%\XeTeXinputencoding "GB2312"
%\documentclass[UTF8, nocolorlinks]{pkuthss}
\documentclass[UTF8]{pkuthss}
\setCJKmainfont[BoldFont={FZHei-B01S}, ItalicFont={FZKai-Z03S}]{FZJPShuSong-Z01S} % FangZhengFonts
\setmainfont{CMU Serif Roman}
%\usepackage{mathptmx}
%\usepackage{unicode-math}
%\setmainfont{Times New Roman}
%\setsansfont{Myriad Pro}
%\setmonofont[Scale=MatchLowercase]{Consolas}
% 若需要按照英文文献在前,中文文献在后排序,请设置“sorting = ecnty”;
% 若需要按照中文文献在前,英文文献在后排序,请设置“sorting = centy”。
\usepackage[backend = biber, style = caspervector, utf8, sorting = none]{biblatex}
\usepackage{amssymb}


% 提供 Verbatim 环境和 \VerbatimInput 命令。
\usepackage{fancyvrb}

% 章节标题
\CTEXsetup[nameformat={\zihao{-3}\bfseries},titleformat={\zihao{-3}\bfseries}]{chapter}
\CTEXsetup[format={\zihao{4}\bfseries\centering}]{section}
\CTEXsetup[format={\zihao{-4}\bfseries}]{subsection}
\CTEXsetup[format={\normalsize\bfseries}]{subsubsection}
% 使被强调的内容为红色。
\newcommand{\myemph}[1]{\emph{#1}}
% pkuthss 文档模版的版本。
\newcommand{\docversion}{v1.4 rc3}
% 设定文档的基本信息。
\pkuthssinfo{
	cthesisname = {本科生毕业论文}, ethesisname = {Undergraduate Thesis},
	ctitle = {对71个Planck冷团块的CO(1-0)成图研究},
	etitle = {CO (1-0) Mapping Study \\of 71 Planck Cold Clumps},
	cauthor = {孟繁一},
	eauthor = {Fanyi Meng},
	studentid = {00946063},
	date = {二\chinese{0}一三年六月},
	school = {元培学院},
	cmajor = {物理学}, emajor = {Physics},
	direction = {天体物理},
	cmentor = {吴月芳\ 教授}, ementor = {Prof.\ Yuefang Wu},
	ckeywords = {恒星形成,分子云,分子谱线},
	ekeywords = {Star Formation,Molecular Cloud,Molecular Spectrum}
}
% 导入参考文献数据库(注意不要省略“.bib”)。
\addbibresource{00946063.bib}
%Following are the substitution of commonly used text in the paper.
\newcommand{\vdag}{(v)^\dagger}
\newcommand{\dd}{{\rm d}}
\newcommand{\umlte}{\frac{M_{LTE}}{M_\odot}}
\newcommand{\coa}{$^{12}$CO }
\newcommand{\Planck}{\emph{Planck} }
\newcommand{\cob}{$^{13}$CO }
\newcommand{\coc}{C$^{18}$O }
\newcommand{\coaa}{$^{12}$CO(1-0) }
\newcommand{\cobb}{$^{13}$CO(1-0) }
\newcommand{\cocc}{C$^{18}$O(1-0) }
\newcommand{\multi}{$\times$}
\newcommand{\vlsr}{${\rm V } _{lsr}$}
\newcommand{\kms}{km s$^{-1}$}
\newcommand{\ta}{$T_{\rm A}$}
\newcommand{\texc}{$T_{\rm {ex}}$ }
\newcommand{\taub}{$\tau _{13}$}
\newcommand{\tauc}{$\tau _{18}$}
\newcommand{\tcoa}{$T_{12}$}
\newcommand{\tcob}{$T_{13}$}
\newcommand{\tcoc}{$T_{18}$}
\newcommand{\nb}{$N_{13}$\ }
\newcommand{\nc}{$N_{18}$\ }
\newcommand{\nhyd}{$N_{\rm H_2}$\ }
\newcommand{\nnhyd}{$n_{\rm H_2}$\ }
\newcommand{\pow}[1]{$\times 10^{#1}$}
\newcommand{\epm}{$\pm$}
\newcommand{\pcms}{$\rm cm^{-2}$}
\newcommand{\sigmath}{$\sigma _{Therm}$\ }
\newcommand{\sigmant}{$\sigma _{NT}$\ }
\newcommand{\sigmatd}{$\sigma _{3D}$\ }
\newcommand{\lp}{lognormal}
\newcommand{\np}{normal{}{}{}}
%Following are command not in AASTeX.
\newcommand{\jj}{{\rm j}}
\newcommand{\uph}{$\phi$($^{\circ}$)}
\newcommand{\apjs}{Astrophysical Journal Supplement}
\newcommand{\apjl}{Astrophysical Journal Letters}
\newcommand{\apj}{Astrophysical Journal}
\newcommand{\aap}{Astronomy and Astrophysics}
\newcommand{\araa}{Annual Review of Astronomy and Astrophysics}
\newcommand{\mnras}{Monthly Notices of the Royal Astronomical Society}
\newcommand{\nat}{Nature}
\newcommand{\aj}{The Astronomical Journal}
\newcommand{\arcsec}{$^{\prime\prime}$}
\newcommand{\arcmin}{$^{\prime}$}
%Following are important numbers used in the text.
\newcommand{\numallsou}{74\ }
\newcommand{\numsou}{71\ }
\newcommand{\numsoutmc}{34\ }
\newcommand{\numsoupmc}{13\ }
\newcommand{\numsoucmc}{24\ }

\newcommand{\numclump}{45\ }

\newcommand{\numcore}{38\ }
\newcommand{\numcoretmc}{19\ }
\newcommand{\numcorepmc}{3\ }
\newcommand{\numcorecmc}{16\ }

\newcommand{\numduecomp}{5\ }
\newcommand{\numcocc}{55\ }
\newcommand{\numcompofcores}{27\ }
\newcommand{\numdiffusecomp}{42\ }
\newcommand{\numvelcomp}{76\ }


\begin{document}
	% 以下为正文之前的部分。
	\frontmatter
	% 自动生成标题页。
	\maketitle

	% 中英文摘要。
\begin{cabstract}

	本文介绍了对 Taurus/Perseus/California 三个分子云复合体中71个 Planck 冷团块气体成分的成图研究。 我们利用中国科学院紫金山天文台青海观测站的13.7 m 毫米波望远镜进行了\coa、\cob 以及 \coc 分子的 $J=1-0$ 转动跃迁谱线观测。全部\numsou 个Planck冷团块都被探测到有\coaa 和\cobb 的辐射,其中55个被探测到有\cocc 的辐射。 通过空间分布和线心速度的验证, 我们证认出 Taurus 分子云复合体(TMC)、Perseus 分子云复合体(PMC)和 California 分子云复合体(CMC)中分别有 \numsoutmc 个、 \numsoucmc 个和 \numsoupmc 个Planck冷团块。在这些冷团块中,共证认出\numcore 个云核(Core), 其中TMC、PMC和CMC分别有\numcoretmc、\numcorepmc 和\numcorecmc 个云核。对于这些云核,我们通过观测数据计算了观测参量:线心速度(\vlsr )、天线温度(\ta )、谱线宽度(FWHM)。 通过这些观测参量我们计算出了各个云核的物理参量。 在局部热动平衡(LTE)假设下应用辐射传能方程,我们得到激发温度(\texc )、气体柱密度(\nhyd )。同时我们还计算了云核的热速度弥散(\sigmath )、非热速度弥散(\sigmant )以及三维速度弥散(\sigmatd )。根据对云核成图的椭圆高斯拟合,我们得到了云核的几何尺度($R$),从而计算出气体数密度(\nnhyd )、LTE质量($M_{LTE}$)、Jeans质量($M_{J}$)和virial质量($M_{vir}$)。

	根据这些物理参量,我们发现84\%的云核有:\sigmant $>$\sigmath ,而TMC中84\%的云核和CMC中69\%的云核的柱密度概率分布函数(PDF)呈对数正态分布(K-S检验中$p>0.05$),表明在我们的样本中湍动的支配地位。大多数(TMC中的90\%、CMC中的60\%)云核的$M_{vir}$和$M_{J}$大于$M_{LTE}$,表明云核无明显的重力塌缩迹象。通过对比此气体观测结果和 Planck 获得的尘埃连续谱辐射数据,我们发现60\%云核的气体温度低于其尘埃温度。同时,CO分子的丰度在正常范围之内。经过查找成协源,90\%的团块被证实无成协物。根据这些计算与分析,我们发现这些Planck团块还没处于恒星形成前,分子云中的致密核未充分演化的早期形态。本研究揭示了它们的物理状态和动力学性质。
\end{cabstract}

\begin{eabstract}

	A mapping study towards \numsou Planck Cold Clumps was made with \coaa, \cobb and \cocc lines, at the 13.7 m telescope of Purple Mountain Observatory. For all the clumps, \coaa and \cobb emissions were detected, while for \numcocc of them, \cocc emissions were detected.  Of the \numsou Clumps, \numsoutmc are in Taurus Molecular Complex, \numsoucmc in California  Molecular Complex and \numsoupmc are in Perseus Molecular Complex. In the \numvelcomp velocity components, \numcore cores are found in \numcompofcores clumps, \numcoretmc cores are in TMC, \numcorecmc in CMC and \numcorepmc are in PMC.
    We acquired the observational parameters such as $V_{lsr}$, $T_{A}$ and FWHM of lines. Physical parameters including \texc, \nhyd, \sigmath, \sigmant,  \sigmatd were calculated.

    We found that 84\% of the cores have \sigmant larger than \sigmath and 84\% cores in TMC and 69\% cores in CMC have \nhyd lognormal probability distribution function. These suggest the dominance of turbulence in our cores. Most cores (90\% in TMC and 69\% in CMC) are found with $M_{vir}$ and $M_J$ larger than $M_{LTE}$, indicating these cores are neither going to collapse nor gravitationally bound. To compare with the dust properties revealed by Planck ECC catalog, we investigated the coupling of gas and dust components. We found 60\% of the cores are with dust temperature higher than gas temperature. The objects associated with our sources are checked, for 90\% of the core, no associated objects are found within distance of 55\arcsec to their centers. These facts suggest that our samples represent the very early stage of prestellar cores. This research reveals the physical properties and kinematic characteristics of molecular cores which are in their early evolutionary stages.

\end{eabstract}

	% 自动生成目录。
	\tableofcontents
	% 以下为正文。
	\mainmatter
\chapter{引言}
	\section{冷暗云(Cold Dark Cloud)}
		略

	\section{Planck团块}

		欧洲航天局于2009年5月发射 Planck空间望远镜,其工作波段为:30、44和70 Hz(以上由LFI接收机实现);100、143、217、353、545和857 Hz(以上由HFI接收机实现)\supercite{2011A&A...536A...1P}。Planck 的早期结果系统地提供了丰富的银河系内冷团块样本:C3PO(Cold Core Catalogue of Planck Objects) 星表包含10,873个冷云核\supercite{2011yCat.8088....0P,2011A&A...536A..23P}。在C3PO表中,915个可靠探测($\rm SNR>15$ 且 $T_{\rm ECC}<14\rm\ K$)的源被归入其子表ECC(Planck Early Release Cold Cores Catalogue)\supercite{2011yCat.8088....0P,2011A&A...536A..23P}。

		对于Planck冷团块中尘埃的物理性质,Planck Collaboration进行了的细致研究,从而揭示了C3PO中的冷团块的柱密度$N_ {\rm H_2} \approx 0.1 \rm \sim \ 1.6 \times 10^{22} \rm cm^{-2}$,并且尘埃温度在10 K到15 K之间\supercite{2011A&A...536A..23P}。这些研究利用三个HFI接收机频率(353、545、857 Hz)测得的流量,同时结合IRAS 3000 GHz波段的数据得出相应结果\supercite{2011A&A...536A..23P,2011A&A...536A..22P}。

		除了Planck Collaboration 对于 Planck 冷团块尘埃成分的研究之外, 对其气体成分的单点观测也已经取得成果。 利用中国科学院紫金山天文台青海观测站的13.7 m毫米波望远镜,Wu et al (2012)完成了对674个ECC源的\coaa 、\cobb  和\cocc  三谱线单点观测,并且得出了其动力学温度($T_k$)、气体柱密度(\nhyd)以及速度弥散($\sigma$)等物理参量\supercite{wu2012gas}。利用同样的设备,Liu et al (2012)等对Orion分子云复合体中的51个Planck冷团块做了\coaa 、\cobb  和\cocc  三谱线成图观测\supercite{LiuTie}。

	\section{研究区域}

		\subsection{Taurus分子云复合体}

			Taurus分子云复合体(Taurus Molecular Complex,下文简称TMC)是典型的小质量恒星形成区, 同时也是距离我们最近($D\approx 140\rm\ pc$)的巨分子云\supercite{1987ApJ...322..706D}。 因此,在分辨率有限的情况下,TMC是理想的小质量恒星形成研究样本。

			对于TMC,人们已经有大量的分子谱线观测结果。针对TMC和Ophuichus中的90个“微小光学不透明体”(small visual opaques),Myers (1983), Myers and Benson (1983) 进行了\cob 和 \coc 的观测,并且得出其中的云核几何尺度为$0.1\sim0.3 \ \rm pc$,质量为$4\sim30\ \rm M_\odot$以及激发温度\texc $\sim$ 10 K\supercite{1983ApJ...264..517M,1983ApJ...266..309M}。在同一系列的成果中,Meyers et al 对云核的亚音速湍动、CO外向流、形态等也进行了研究\supercite{1983ApJ...270..105M,1988ApJ...324..907M,1991ApJ...376..561M}。

			在TMC中,人们发现了丰富的气体动力学过程和多个演化阶段的特征。分子外向流首次在大质量恒星形成区被发现,但由于距离过远其并不能被很好地被分辨\supercite{1976ApJ...209L.137Z}。而之后在TMC中L1551被证认为有分子外向流,并且红瓣(red-lobe)和蓝瓣(blue-lobe)能够良好地被分辨\supercite{1980ApJ...239L..17S}。截至2003年2月发现的400个分子外向流中,有11\% 在TMC中\supercite{2004A&A...426..503W}。不仅分子外向流,人们在TMC中也发现了丰富的的分子内向流:例如对T Tau 双星系统\supercite{1994ApJ...425L..45V}和L 1544的研究\supercite{1998ApJ...504..900T}。和恒星形成密切相关的盘状结构也在TMC中被证认出来:例如在HL Tau周围发现的盘状结构\supercite{1991ApJ...382L..31S}以及DM Tau\supercite{1995ApJ...453..384S}。综上,TMC包含从class 0到class III全部恒星形成阶段的源。这些不同演化阶段的源的特征是恒星形成理论,例如Shu et al (1987) 提出的“恒星形成四个阶段”理论\supercite{shu1987star}的观测基础。

		\subsection{Perseus分子云复合体}

			Perseus分子云复合体(Persues Molecular Complex,下文简称PMC)同样是著名的恒星形成区, 其中有中等质量原恒星被证认\supercite{2010A&A...512A..67L}。 一般认为,PMC的情形介于以TMC为代表的小质量恒星形成区和以Orion巨分子云为代表的大质量恒星形成区之间\supercite{2010ApJ...711..655J}。对于PMC,人们亦有大量研究。

			结合亚毫米和中红外资料,在PMC中已有49个深嵌埋(deeply embedded)年轻星体(young stellar objects,下文简称YSO)被证认\supercite{2007ApJ...656..293J}。 而在毫米波段,针对PMC人们也已经进行过多次CO观测\supercite{1979ApJ...233..163S,1999ApJ...525..318P, 2005A&A...440..151H}。

			PMC的距离为235$\pm$18 pc\supercite{2010A&A...512A..67L}。

		\subsection{California分子云复合体}

			有别于TMC和PMC,在2009年之前,California分子云复合体(California Molecular Complex,下文简称CMC) 并没有获得足够的重视。对于CMC的开创性工作是Lada et al在2009年作出的,其得出了CMC的距离为$450\pm23\rm\ pc$,空间延伸达80 pc,总质量约为 $10^5 M_\odot$\supercite{2009ApJ...703...52L}。CMC有着和Orion巨分子云相似的距离、质量和形态,但是其恒星形成活跃程度对比Orion巨分子云要低得多\supercite{2009ApJ...703...52L,2010A&A...512A..67L}。

			近期,一项对于CMC区域的成图研究给出了含有78个源的星表\supercite{2013ApJ...764..133H}。其中有60个“致密源”在70 $\mu$m的波段上被证认,所用数据来自Herschel的PACS 和 SPIRE设备\supercite{2013ApJ...764..133H}。而另外18个“冷致密源”则是用的Bolocam 1.1 mm 波段的数据\supercite{2013ApJ...764..133H}。

			由于CMC展示出了相对年轻的演化特征,其可以作为研究冷云核早期演化的理想样本。除了上述的研究,我们仍然迫切需要对CMC做更高分辨率的毫米波观测,以更全面而深入地揭示其中冷云核的物理特性。

	\section{研究目标}
\chapter{观测和数据处理}
	\section{望远镜参数}
	\section{数据处理方法}
\chapter{观测结果}
	\section{观测参量}
	\section{物理参量}
\chapter{分析与讨论}
	\section{谱线轮廓}
	\section{云核内的湍动}
	\section{云核的重力稳定性}
	\section{气体-尘埃耦合}
	\section{CO丰度}
	\section{云核的成协情况}
	% 正文中的附录部分。
	\appendix
	% 排版参考文献列表。
	\printbibliography[
		% 使“参考文献”出现在目录中;如果同时要使参考文献列表参与章节编号,
		% 可将“bibintoc”改为“bibnumbered”。
		heading = bibintoc,
		% 单独设定排序方案。此设定会局部覆盖之前的全局设置。
		% 注:只有同时使用 2.x 或之后版本的 biblatex 和相应兼容版本的 biber,
		% 才能对每个 \printbibliography 命令采用不同的排序方案,
		% 否则只能在导入 biblatex 宏包时就(全局)指定排序方案。
		%sorting = none
	]
	% 各附录。
\iffalse
\chapter{气体核的中心谱线}

	\begin{figure}[ht]
	\includegraphics[totalheight=72mm]{{SpatiaDist_Velocity_Overlay}.eps}
	\caption{The spatial distribution of the sources overlap on the contours of \cob data from \cite{2001ApJ...547..792D}, the color of each dot represent their centroid velocities.
	The dashed lines show the boundaries separating three regions: TMC, PMC and CMC, which is according to \cite{2010A&A...512A..67L}.
	\label{Fig.SpatialDistribution}}
	\end{figure}


	\begin{figure}[ht]
	\centering
	\caption{Spectra at emission peaks. For each plot, 3 lines of \coaa \cobb and \cocc are colored red, green and blue respectively. The offsets of the positions at where the spectra where observed are also shown, the unit is arcmin.\label{Fig.Spectra}}
	\includegraphics[width=48mm,height=32mm]{{G153.34-08.00C1_spec}.ps}
	\includegraphics[width=48mm,height=32mm]{{G153.34-08.00C2_spec}.ps}
	\includegraphics[width=48mm,height=32mm]{{G154.68-15.34C1_spec}.ps}\\
	\includegraphics[width=48mm,height=32mm]{{G155.52-08.93_spec}.ps}
	\includegraphics[width=48mm,height=32mm]{{G156.53-08.63_spec}.ps}
	\includegraphics[width=48mm,height=32mm]{{G156.90-08.49_spec}.ps}\\
	\includegraphics[width=48mm,height=32mm]{{G156.92-09.72C1_spec}.ps}
	\includegraphics[width=48mm,height=32mm]{{G157.10-08.70_spec}.ps}
	\includegraphics[width=48mm,height=32mm]{{G157.12-11.56C1_spec}.ps}\\
	\includegraphics[width=48mm,height=32mm]{{G157.12-11.56C2_spec}.ps}
	\includegraphics[width=48mm,height=32mm]{{G157.12-11.56C3_spec}.ps}
	\includegraphics[width=48mm,height=32mm]{{G157.19-08.81_spec}.ps}\\
	\end{figure}\clearpage
	%%%%%%%%%%%%%%%%%%%%%%%%%%%%%%%

	%%%%%%%%%%%%%%%%%%%%%%%%%%%%%%%
	\begin{figure}[ht]
	\includegraphics[width=48mm,height=32mm]{{G157.58-08.89_spec}.ps}
	\includegraphics[width=48mm,height=32mm]{{G157.60-12.17a_spec}.ps}
	\includegraphics[width=48mm,height=32mm]{{G157.60-12.17bC1_spec}.ps}\\
	\includegraphics[width=48mm,height=32mm]{{G157.60-12.17bC2_spec}.ps}
	\includegraphics[width=48mm,height=32mm]{{G157.91-08.23C1_spec}.ps}
	\includegraphics[width=48mm,height=32mm]{{G158.20-20.28_spec}.ps}\\
	\includegraphics[width=48mm,height=32mm]{{G158.22-20.14_spec}.ps}
	\includegraphics[width=48mm,height=32mm]{{G158.24-21.80_spec}.ps}
	\includegraphics[width=48mm,height=32mm]{{G158.37-20.72_spec}.ps}\\
	\includegraphics[width=48mm,height=32mm]{{G158.40-21.86_spec}.ps}
	\includegraphics[width=48mm,height=32mm]{{G159.01-08.46_spec}.ps}
	\includegraphics[width=48mm,height=32mm]{{G159.03-08.32_spec}.ps}\\
	\includegraphics[width=48mm,height=32mm]{{G159.14-08.76_spec}.ps}
	\includegraphics[width=48mm,height=32mm]{{G159.16-05.58_spec}.ps}
	\includegraphics[width=48mm,height=32mm]{{G159.21-20.12C1_spec}.ps}\\
	\includegraphics[width=48mm,height=32mm]{{G159.65-19.68a_spec}.ps}
	\includegraphics[width=48mm,height=32mm]{{G159.65-19.68b_spec}.ps}
	\includegraphics[width=48mm,height=32mm]{{G159.67-05.71aC1_spec}.ps}\\
	\end{figure}\clearpage
	\begin{figure}[ht]
	\includegraphics[width=48mm,height=32mm]{{G159.67-05.71aC2_spec}.ps}
	\includegraphics[width=48mm,height=32mm]{{G159.67-05.71bC1_spec}.ps}
	\includegraphics[width=48mm,height=32mm]{{G159.82-10.48C1_spec}.ps}\\
	\includegraphics[width=48mm,height=32mm]{{G159.82-10.48C2_spec}.ps}
	\includegraphics[width=48mm,height=32mm]{{G160.35-06.37_spec}.ps}
	\includegraphics[width=48mm,height=32mm]{{G160.51-16.84_spec}.ps}\\
	\includegraphics[width=48mm,height=32mm]{{G160.51-17.07_spec}.ps}
	\includegraphics[width=48mm,height=32mm]{{G160.53-09.84_spec}.ps}
	\includegraphics[width=48mm,height=32mm]{{G160.53-19.72C1_spec}.ps}\\
	\includegraphics[width=48mm,height=32mm]{{G160.62-16.70_spec}.ps}
	\includegraphics[width=48mm,height=32mm]{{G161.08-21.78_spec}.ps}
	\includegraphics[width=48mm,height=32mm]{{G161.21-08.72_spec}.ps}\\
	\includegraphics[width=48mm,height=32mm]{{G162.24-09.04C1_spec}.ps}
	\includegraphics[width=48mm,height=32mm]{{G162.46-08.70_spec}.ps}
	\includegraphics[width=48mm,height=32mm]{{G162.86-05.24_spec}.ps}\\
	\includegraphics[width=48mm,height=32mm]{{G163.32-08.42_spec}.ps}
	\includegraphics[width=48mm,height=32mm]{{G164.13-08.85C1_spec}.ps}
	\includegraphics[width=48mm,height=32mm]{{G164.57-24.48C1_spec}.ps}\\
	\end{figure}\clearpage
	\begin{figure}[ht]
	\includegraphics[width=48mm,height=32mm]{{G164.57-24.48C2_spec}.ps}
	\includegraphics[width=48mm,height=32mm]{{G164.57-24.48C3_spec}.ps}
	\includegraphics[width=48mm,height=32mm]{{G164.75-24.19C1_spec}.ps}\\
	\includegraphics[width=48mm,height=32mm]{{G164.75-24.19C2_spec}.ps}
	\includegraphics[width=48mm,height=32mm]{{G164.81-05.67C1_spec}.ps}
	\includegraphics[width=48mm,height=32mm]{{G164.92-12.65C1_spec}.ps}\\
	\includegraphics[width=48mm,height=32mm]{{G164.92-12.65C2_spec}.ps}
	\includegraphics[width=48mm,height=32mm]{{G164.92-12.65C3_spec}.ps}
	\includegraphics[width=48mm,height=32mm]{{G166.99-15.34C1_spec}.ps}\\
	\includegraphics[width=48mm,height=32mm]{{G168.00-15.69_spec}.ps}
	\includegraphics[width=48mm,height=32mm]{{G168.13-16.39a_spec}.ps}
	\includegraphics[width=48mm,height=32mm]{{G168.13-16.39b_spec}.ps}\\
	\includegraphics[width=48mm,height=32mm]{{G169.76-16.15a_spec}.ps}
	\includegraphics[width=48mm,height=32mm]{{G169.76-16.15b_spec}.ps}
	\includegraphics[width=48mm,height=32mm]{{G169.82-19.39_spec}.ps}\\
	\includegraphics[width=48mm,height=32mm]{{G169.98-18.97_spec}.ps}
	\includegraphics[width=48mm,height=32mm]{{G170.13-16.06a_spec}.ps}
	\includegraphics[width=48mm,height=32mm]{{G170.13-16.06b_spec}.ps}\\
	\end{figure}\clearpage
	\begin{figure}[ht]
	\includegraphics[width=48mm,height=32mm]{{G170.26-16.02_spec}.ps}
	\includegraphics[width=48mm,height=32mm]{{G170.57-07.85_spec}.ps}
	\includegraphics[width=48mm,height=32mm]{{G170.88-10.92_spec}.ps}\\
	\includegraphics[width=48mm,height=32mm]{{G171.43-17.36_spec}.ps}
	\includegraphics[width=48mm,height=32mm]{{G172.13-16.93_spec}.ps}
	\includegraphics[width=48mm,height=32mm]{{G172.83-05.17C1_spec}.ps}\\
	\includegraphics[width=48mm,height=32mm]{{G172.85-14.74_spec}.ps}
	\includegraphics[width=48mm,height=32mm]{{G172.92-16.74a_spec}.ps}
	\includegraphics[width=48mm,height=32mm]{{G172.92-16.74b_spec}.ps}\\
	\includegraphics[width=48mm,height=32mm]{{G172.94-05.49C1_spec}.ps}
	\includegraphics[width=48mm,height=32mm]{{G173.07-17.89C1_spec}.ps}
	\includegraphics[width=48mm,height=32mm]{{G173.07-17.89C2_spec}.ps}\\
	\includegraphics[width=48mm,height=32mm]{{G173.45-05.43_spec}.ps}
	\includegraphics[width=48mm,height=32mm]{{G173.60-17.89C1_spec}.ps}
	\includegraphics[width=48mm,height=32mm]{{G173.60-17.89C2_spec}.ps}\\
	\includegraphics[width=48mm,height=32mm]{{G173.91-16.25_spec}.ps}
	\includegraphics[width=48mm,height=32mm]{{G174.06-15.81_spec}.ps}
	\includegraphics[width=48mm,height=32mm]{{G174.39-13.43_spec}.ps}\\
	\end{figure}\clearpage
	\begin{figure}[ht]
	\includegraphics[width=48mm,height=32mm]{{G174.70-15.48C1_spec}.ps}
	\includegraphics[width=48mm,height=32mm]{{G175.16-16.74_spec}.ps}
	\includegraphics[width=48mm,height=32mm]{{G175.97-20.38C1_spec}.ps}\\
	\includegraphics[width=48mm,height=32mm]{{G176.46-21.10_spec}.ps}
	\includegraphics[width=48mm,height=32mm]{{G177.60-20.58_spec}.ps}
	\includegraphics[width=48mm,height=32mm]{{G177.89-20.16a_spec}.ps}\\
	\includegraphics[width=48mm,height=32mm]{{G177.89-20.16b_spec}.ps}
	\includegraphics[width=48mm,height=32mm]{{G177.97-09.72_spec}.ps}\\
	\end{figure}\clearpage
	%%%%%%%%%%%%%%%%%%%%%%%%%%%%%%%%%%%%%%%%%%
\chapter{气体核的成图结果}

	%%%%%%%%%%%%%%%%%%%%%%%%%%%%%%%%%%%%%%%%%%%
	\begin{figure}[ht]
	\caption{Maps. \cobb integral intensity contours over \coaa grey-scale. The contour levels are 30\% to 90\%, with step of 10\%. Yellow and blue (variety of colors to ensure distinction against background)  symbols represent associated objects: $\times$(X-ray object),  $\triangle$(radio, \textsc{H i}, Maser), $\Diamond$(IR object), or some denoted as text abbreviations: HH(Herbig-Haro object), YSO(Young Stellar Object), etc. The types and locations of  associated objects are according to Aladin application in SIMBAD database and the meaning of symbols and abbreviations are available there as well.\label{Maps}}
	\centering
	\begin{minipage}[c]{0.3\textwidth}  \centering   \includegraphics[width=48mm,height=42mm]{{G153.34-08.00}.eps}   \end{minipage}
	\begin{minipage}[c]{0.3\textwidth}  \centering   \includegraphics[width=48mm,height=42mm]{{G154.68-15.34}.eps}   \end{minipage}
	\begin{minipage}[c]{0.3\textwidth}  \centering   \includegraphics[width=48mm,height=42mm]{{G155.52-08.93}.eps}   \end{minipage}\\
	\begin{minipage}[c]{0.3\textwidth}  \centering   \includegraphics[width=48mm,height=42mm]{{G156.53-08.63}.eps}   \end{minipage}
	\begin{minipage}[c]{0.3\textwidth}  \centering   \includegraphics[width=48mm,height=42mm]{{G156.90-08.49}.eps}   \end{minipage}
	\begin{minipage}[c]{0.3\textwidth}  \centering   \includegraphics[width=48mm,height=42mm]{{G156.92-09.72}.eps}   \end{minipage} \\
	\begin{minipage}[c]{0.3\textwidth}  \centering   \includegraphics[width=48mm,height=42mm]{{G157.10-08.70}.eps}   \end{minipage}
	\begin{minipage}[c]{0.3\textwidth}  \centering   \includegraphics[width=48mm,height=42mm]{{G157.12-11.56}.eps}   \end{minipage}
	\begin{minipage}[c]{0.3\textwidth}  \centering   \includegraphics[width=48mm,height=42mm]{{G157.19-08.81}.eps}   \end{minipage} \\
	\end{figure}\clearpage
	%%%%%%%%%%%%%%%%%%%%%%%%%%%%%%%%
	%%%%%%%%%%%%%%%%%%%%%%%%%%%%%%%%
	\begin{figure}[ht]
	\centering
	\begin{minipage}[c]{0.3\textwidth}  \centering   \includegraphics[width=48mm,height=42mm]{{G157.58-08.89}.eps}   \end{minipage}
	\begin{minipage}[c]{0.3\textwidth}  \centering   \includegraphics[width=48mm,height=42mm]{{G157.60-12.17}.eps}   \end{minipage}
	\begin{minipage}[c]{0.3\textwidth}  \centering   \includegraphics[width=48mm,height=42mm]{{G157.60-12.17b}.eps}   \end{minipage}\\
	\begin{minipage}[c]{0.3\textwidth}  \centering   \includegraphics[width=48mm,height=42mm]{{G157.91-08.23}.eps}   \end{minipage}
	\begin{minipage}[c]{0.3\textwidth}  \centering   \includegraphics[width=48mm,height=42mm]{{G158.20-20.28}.eps}   \end{minipage}
	\begin{minipage}[c]{0.3\textwidth}  \centering   \includegraphics[width=48mm,height=42mm]{{G158.22-20.14}.eps}   \end{minipage} \\
	\begin{minipage}[c]{0.3\textwidth}  \centering   \includegraphics[width=48mm,height=42mm]{{G158.24-21.80}.eps}   \end{minipage}
	\begin{minipage}[c]{0.3\textwidth}  \centering   \includegraphics[width=48mm,height=42mm]{{G158.37-20.72}.eps}   \end{minipage}
	\begin{minipage}[c]{0.3\textwidth}  \centering   \includegraphics[width=48mm,height=42mm]{{G158.40-21.86}.eps}   \end{minipage} \\
	\begin{minipage}[c]{0.3\textwidth}  \centering   \includegraphics[width=48mm,height=42mm]{{G159.01-08.46}.eps}   \end{minipage}
	\begin{minipage}[c]{0.3\textwidth}  \centering   \includegraphics[width=48mm,height=42mm]{{G159.03-08.32}.eps}   \end{minipage}
	\begin{minipage}[c]{0.3\textwidth}  \centering   \includegraphics[width=48mm,height=42mm]{{G159.14-08.76}.eps}   \end{minipage} \\
	\begin{minipage}[c]{0.3\textwidth}  \centering   \includegraphics[width=48mm,height=42mm]{{G159.16-05.58}.eps}   \end{minipage}
	\begin{minipage}[c]{0.3\textwidth}  \centering   \includegraphics[width=48mm,height=42mm]{{G159.21-20.12}.eps}   \end{minipage}
	\begin{minipage}[c]{0.3\textwidth}  \centering   \includegraphics[width=48mm,height=42mm]{{G159.65-19.68}.eps}   \end{minipage} \\
	\end{figure}\clearpage
	\begin{figure}[ht]
	\centering
	\begin{minipage}[c]{0.3\textwidth}  \centering   \includegraphics[width=48mm,height=42mm]{{G159.65-19.68b}.eps}   \end{minipage}
	\begin{minipage}[c]{0.3\textwidth}  \centering   \includegraphics[width=48mm,height=42mm]{{G159.67-05.71}.eps}   \end{minipage}
	\begin{minipage}[c]{0.3\textwidth}  \centering   \includegraphics[width=48mm,height=42mm]{{G159.82-10.48}.eps}   \end{minipage} \\
	\begin{minipage}[c]{0.3\textwidth}  \centering   \includegraphics[width=48mm,height=42mm]{{G160.35-06.37}.eps}   \end{minipage}
	\begin{minipage}[c]{0.3\textwidth}  \centering   \includegraphics[width=48mm,height=42mm]{{G160.51-16.84}.eps}   \end{minipage}
	\begin{minipage}[c]{0.3\textwidth}  \centering   \includegraphics[width=48mm,height=42mm]{{G160.51-17.07}.eps}   \end{minipage} \\
	\begin{minipage}[c]{0.3\textwidth}  \centering   \includegraphics[width=48mm,height=42mm]{{G160.53-09.84}.eps}   \end{minipage}
	\begin{minipage}[c]{0.3\textwidth}  \centering   \includegraphics[width=48mm,height=42mm]{{G160.53-19.72}.eps}   \end{minipage}
	\begin{minipage}[c]{0.3\textwidth}  \centering   \includegraphics[width=48mm,height=42mm]{{G160.62-16.70}.eps}   \end{minipage} \\
	\begin{minipage}[c]{0.3\textwidth}  \centering   \includegraphics[width=48mm,height=42mm]{{G161.08-21.78}.eps}   \end{minipage}
	\begin{minipage}[c]{0.3\textwidth}  \centering   \includegraphics[width=48mm,height=42mm]{{G161.21-08.72}.eps}   \end{minipage}
	\begin{minipage}[c]{0.3\textwidth}  \centering   \includegraphics[width=48mm,height=42mm]{{G162.24-09.04}.eps}   \end{minipage} \\
	\begin{minipage}[c]{0.3\textwidth}  \centering   \includegraphics[width=48mm,height=42mm]{{G162.46-08.70}.eps}   \end{minipage}
	\begin{minipage}[c]{0.3\textwidth}  \centering   \includegraphics[width=48mm,height=42mm]{{G162.86-05.24}.eps}   \end{minipage}
	\begin{minipage}[c]{0.3\textwidth}  \centering   \includegraphics[width=48mm,height=42mm]{{G163.32-08.42}.eps}   \end{minipage} \\
	\end{figure}\clearpage
	\begin{figure}[ht]
	\centering
	\begin{minipage}[c]{0.3\textwidth}  \centering   \includegraphics[width=48mm,height=42mm]{{G164.13-08.85}.eps}   \end{minipage}
	\begin{minipage}[c]{0.3\textwidth}  \centering   \includegraphics[width=48mm,height=42mm]{{G164.57-24.48}.eps}   \end{minipage}
	\begin{minipage}[c]{0.3\textwidth}  \centering   \includegraphics[width=48mm,height=42mm]{{G164.75-24.19}.eps}   \end{minipage} \\
	\begin{minipage}[c]{0.3\textwidth}  \centering   \includegraphics[width=48mm,height=42mm]{{G164.81-05.67}.eps}   \end{minipage}
	\begin{minipage}[c]{0.3\textwidth}  \centering   \includegraphics[width=48mm,height=42mm]{{G164.92-12.65}.eps}   \end{minipage}
	\begin{minipage}[c]{0.3\textwidth}  \centering   \includegraphics[width=48mm,height=42mm]{{G166.99-15.34}.eps}   \end{minipage} \\
	\begin{minipage}[c]{0.3\textwidth}  \centering   \includegraphics[width=48mm,height=42mm]{{G168.00-15.69}.eps}   \end{minipage}
	\begin{minipage}[c]{0.3\textwidth}  \centering   \includegraphics[width=48mm,height=42mm]{{G168.13-16.39}.eps}   \end{minipage}
	\begin{minipage}[c]{0.3\textwidth}  \centering   \includegraphics[width=48mm,height=42mm]{{G169.76-16.15}.eps}   \end{minipage} \\
	\begin{minipage}[c]{0.3\textwidth}  \centering   \includegraphics[width=48mm,height=42mm]{{G169.76-16.15b}.eps}   \end{minipage}
	\begin{minipage}[c]{0.3\textwidth}  \centering   \includegraphics[width=48mm,height=42mm]{{G169.82-19.39}.eps}   \end{minipage}
	\begin{minipage}[c]{0.3\textwidth}  \centering   \includegraphics[width=48mm,height=42mm]{{G169.98-18.97}.eps}   \end{minipage} \\
	\begin{minipage}[c]{0.3\textwidth}  \centering   \includegraphics[width=48mm,height=42mm]{{G170.13-16.06}.eps}   \end{minipage}
	\begin{minipage}[c]{0.3\textwidth}  \centering   \includegraphics[width=48mm,height=42mm]{{G170.26-16.02}.eps}   \end{minipage}
	\begin{minipage}[c]{0.3\textwidth}  \centering   \includegraphics[width=48mm,height=42mm]{{G170.57-07.85}.eps}   \end{minipage} \\
	\end{figure}\clearpage
	\begin{figure}[ht]
	\centering
	\begin{minipage}[c]{0.3\textwidth}  \centering   \includegraphics[width=48mm,height=42mm]{{G170.88-10.92}.eps}   \end{minipage}
	\begin{minipage}[c]{0.3\textwidth}  \centering   \includegraphics[width=48mm,height=42mm]{{G171.43-17.36}.eps}   \end{minipage}
	\begin{minipage}[c]{0.3\textwidth}  \centering   \includegraphics[width=48mm,height=42mm]{{G172.13-16.93}.eps}   \end{minipage} \\
	\begin{minipage}[c]{0.3\textwidth}  \centering   \includegraphics[width=48mm,height=42mm]{{G172.83-05.17}.eps}   \end{minipage}
	\begin{minipage}[c]{0.3\textwidth}  \centering   \includegraphics[width=48mm,height=42mm]{{G172.85-14.74}.eps}   \end{minipage}
	\begin{minipage}[c]{0.3\textwidth}  \centering   \includegraphics[width=48mm,height=42mm]{{G172.92-16.74}.eps}   \end{minipage} \\
	\begin{minipage}[c]{0.3\textwidth}  \centering   \includegraphics[width=48mm,height=42mm]{{G172.92-16.74b}.eps}   \end{minipage}
	\begin{minipage}[c]{0.3\textwidth}  \centering   \includegraphics[width=48mm,height=42mm]{{G172.94-05.49}.eps}   \end{minipage}
	\begin{minipage}[c]{0.3\textwidth}  \centering   \includegraphics[width=48mm,height=42mm]{{G173.07-17.89}.eps}   \end{minipage} \\
	\begin{minipage}[c]{0.3\textwidth}  \centering   \includegraphics[width=48mm,height=42mm]{{G173.45-05.43}.eps}   \end{minipage}
	\begin{minipage}[c]{0.3\textwidth}  \centering   \includegraphics[width=48mm,height=42mm]{{G173.60-17.89}.eps}   \end{minipage}
	\begin{minipage}[c]{0.3\textwidth}  \centering   \includegraphics[width=48mm,height=42mm]{{G173.91-16.25}.eps}   \end{minipage} \\
	\begin{minipage}[c]{0.3\textwidth}  \centering   \includegraphics[width=48mm,height=42mm]{{G174.06-15.81}.eps}   \end{minipage}
	\begin{minipage}[c]{0.3\textwidth}  \centering   \includegraphics[width=48mm,height=42mm]{{G174.39-13.43}.eps}   \end{minipage}
	\begin{minipage}[c]{0.3\textwidth}  \centering   \includegraphics[width=48mm,height=42mm]{{G174.70-15.48}.eps}   \end{minipage} \\
	\end{figure}\clearpage
	\begin{figure}[ht]
	\centering
	\begin{minipage}[c]{0.3\textwidth}  \centering   \includegraphics[width=48mm,height=42mm]{{G175.16-16.74}.eps}   \end{minipage}
	\begin{minipage}[c]{0.3\textwidth}  \centering   \includegraphics[width=48mm,height=42mm]{{G175.97-20.38}.eps}   \end{minipage}
	\begin{minipage}[c]{0.3\textwidth}  \centering   \includegraphics[width=48mm,height=42mm]{{G176.46-21.10}.eps}   \end{minipage} \\
	\begin{minipage}[c]{0.3\textwidth}  \centering   \includegraphics[width=48mm,height=42mm]{{G177.60-20.58}.eps}   \end{minipage}
	\begin{minipage}[c]{0.3\textwidth}  \centering   \includegraphics[width=48mm,height=42mm]{{G177.89-20.16}.eps}   \end{minipage}
	\begin{minipage}[c]{0.3\textwidth}  \centering   \includegraphics[width=48mm,height=42mm]{{G177.89-20.16b}.eps}   \end{minipage}\\
	\begin{minipage}[c]{0.3\textwidth}  \centering   \includegraphics[width=48mm,height=42mm]{{G177.97-09.72}.eps}   \end{minipage}
	\end{figure}\clearpage
	%%%%%%%%%END%%%%%%%%%%%%%%%%%%%%%%%%%%%%%%%%%%%%%%%%%%%%%%
	%%%%%%%%%%%%%%%%%%%%%%%%%%%%%%%%%%%%%%
	%%%%%%%%%%%%%%%%%%%%%%%%%%%%%%%%%%%%%%


	\begin{figure}[ht]
	\includegraphics[totalheight=82mm]{{1813}.eps}
	\caption{The distribution of ${N_{\rm H_2}(\rm C^{18}O)}/{N_{\rm H_2}(\rm ^{13}CO)}$ of 17 cores with \cocc detection. As we have ruled out cores in PMC from discussion, the two cores in PMC that detected with \cocc emission are excluded. \label{Fig.1813}}.
	\end{figure}


	\begin{figure}[ht]
	\includegraphics[totalheight=52mm]{{pv_G164.75-24.19}.eps}
	\includegraphics[totalheight=52mm]{{pv_G173.07-17.89}.eps}
	\caption{P-V diagram of two cores. Left is taken from RA offset 0', DEC offset from -5' to 5'; Right is taken from RA offset -0.5', DEC offset from -5' to 5'. The positions are plotted over the contour maps of the two cores in Figure \ref{Maps}, as yellow solid lines. \label{Fig. PV_diagram}}.
	\end{figure}

	\begin{figure}[ht]
	\includegraphics[totalheight=120mm]{{VelocityComparison}.eps}
	\caption{The distribution of \sigmath, \sigmant and \sigmath/\sigmant.\label{Fig.SigmaTH/SigmaNT}}
	\end{figure}

	\begin{figure}[ht]
	\includegraphics[totalheight=62mm]{{PDF_G173.60-17.89C2_final}.eps}
	\includegraphics[totalheight=62mm]{{PDF_G159.21-20.12C1_final}.eps}
	\includegraphics[totalheight=62mm]{{PDF_G157.60-12.17bC2_final}.eps}
	\caption{The examples of \nhyd PDF of cores in three regions, with lognormal fit.\label{Fig. PDF}}.
	\end{figure}

	\begin{figure}[ht]
	\includegraphics[totalheight=62mm]{{M_vir_tmc}.eps}
	\includegraphics[totalheight=62mm]{{M_j_tmc}.eps}
	\addtocounter{figure}{1}
	%\figurenum{\arabic{figure}a}
	\caption{Left panel: The correlation between $M_{LTE}$ and $M_{vir}$ in TMC. Right panel: The correlation between $M_{LTE}$ and $M_{J}$ in TMC. The red line is the power law fitting function, while the blue dotted line is the $M_{LTE}=M_{vir}$line, which also is the critical line for gravitational bound state (below the line).\label{mass_tmc}}
	\end{figure}

	\begin{figure}[ht]
	\includegraphics[totalheight=62mm]{{M_vir_cmc}.eps}
	\includegraphics[totalheight=62mm]{{M_j_cmc}.eps}
	%\figurenum{\arabic{figure}b}
	\caption{Left panel: The correlation between $M_{LTE}$ and $M_{vir}$ in CMC. Right panel: The correlation between $M_{LTE}$ and $M_{J}$ in CMC. The red line is the power law fitting function, while the blue dotted line is the $M_{LTE}=M_{vir}$line, which also is the critical line for gravitational bound state (below the line). \label{mass_cmc}}
	\end{figure}

	\begin{figure}[ht]
	\includegraphics[totalheight=62mm]{{Gas-Dust_EB_Core}.eps}
	\caption{$T_D-T_K$ relationship, the $T_K$ are the \texc of our cores while the $T_D$ are from \cite{2011yCat.8088....0P}.\label{Fig.GasDust}}
	\end{figure}
\fi
\chapter{导出参量的计算}
\chapter{本文中术语的中英对照表}
% 以下为正文之后的部分。
\backmatter
% 致谢。
% vim:ts=4:sw=4
%
% Copyright (c) 2008-2009 solvethis
% Copyright (c) 2010-2012 Casper Ti. Vector
% Public domain.

\chapter{致谢}

首先,由衷地感谢我的导师,北京大学天文学系吴月芳教授。我大学二年级伊始,她的谆谆教诲和辛勤辅导引导我开始了天体物理学的研究。近三年来,吴教授一方面细致耐心地指导我的学习和科研,同时也亲切地关心我生活的各个方面。在我本科期间,吴老师支持并资助我三次赴中国科学院紫金山天文台青海观测站进行观测实习,并且鼓励我参与“\emph{Planck}冷团块的气体成分观测”这样意义重大的课题,同时耐心指导我独立完成本文展示的这部分研究。吴教授不仅传授给我天体物理学的思想、知识和方法,她严谨的治学态度和实事求是、追求卓越的科学精神亦将使我终生受用。

感谢同一课题组刘铁的指导和帮助。2011年1月,他在青海观测站辅导我掌握了基本的观测和数据处理方法。之后又不断就研究中的问题给予我耐心的指导。本文中的数据处理用的程序正是基于他编写的脚本而修改得到的。

感谢中国科学院紫金山天文台青海观测站的巨炳刚、孙继先和逯登荣等老师、工程师和全体工作人员。他们的辛勤工作和耐心指导使得我们的观测得以顺利进行,同时他们的细致讲解使得我对射电天文观测有了基本的了解。

感谢天文学系和KIAA的各位老师,他们所讲授的课程和组织的报告与讨论使我有了从事本研究的知识基础和技能储备。特别是:徐仁新老师讲授的《天体物理》,刘富坤老师与刘晓为老师讲授的《恒星大气与天体光谱》,黄茂海老师、卢方军老师和张华伟老师讲授的《天文技术与方法》以及天文学习2907教室丰富多彩的“午餐报告”和KIAA举办的“本科生学术讨论会”,这些课程和讨论会让我获得了知识和技能,也使我由衷地感受到天文学带来的快乐。

感谢同一课题组的任致远、王科、孙宁晨、刘博洋、王文慧、龙凤以及罗连通。他们对我的帮助和与我进行的讨论使我受益匪浅。

篇幅所限,在此对其余所有对本研究作出了贡献的个人和组织表示感谢。

% 原创性声明和使用授权说明。
\include{include/originauth}
\end{document}


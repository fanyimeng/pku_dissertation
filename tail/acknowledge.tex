% vim:ts=4:sw=4
%
% Copyright (c) 2008-2009 solvethis
% Copyright (c) 2010-2012 Casper Ti. Vector
% Public domain.

\chapter{致谢}

首先,由衷地感谢我的导师,北京大学天文学系吴月芳教授。我大学二年级伊始,她的谆谆教诲和辛勤辅导引导我开始了天体物理学的研究。近三年来,吴教授一方面细致耐心地指导我的学习和科研,同时也亲切地关心我生活的各个方面。在我本科期间,吴老师支持并资助我三次赴中国科学院紫金山天文台青海观测站进行观测实习,并且鼓励我参与“\emph{Planck}冷团块的气体成分观测”这样意义重大的课题,同时耐心指导我独立完成本文展示的这部分研究。吴教授不仅传授给我天体物理学的思想、知识和方法,她严谨的治学态度和实事求是、追求卓越的科学精神亦将使我终生受用。

感谢同一课题组刘铁的指导和帮助。2011年1月,他在青海观测站辅导我掌握了基本的观测和数据处理方法。之后又不断就研究中的问题给予我耐心的指导。本文中的数据处理用的程序正是基于他编写的脚本而修改得到的。

感谢中国科学院紫金山天文台青海观测站的巨炳刚、孙继先和逯登荣等老师、工程师和全体工作人员。他们的辛勤工作和耐心指导使得我们的观测得以顺利进行,同时他们的细致讲解使得我对射电天文观测有了基本的了解。

感谢天文学系和KIAA的各位老师,他们所讲授的课程和组织的报告与讨论使我有了从事本研究的知识基础和技能储备。特别是:徐仁新老师讲授的《天体物理》,刘富坤老师与刘晓为老师讲授的《恒星大气与天体光谱》,黄茂海老师、卢方军老师和张华伟老师讲授的《天文技术与方法》以及天文学习2907教室丰富多彩的“午餐报告”和KIAA举办的“本科生学术讨论会”,这些课程和讨论会让我获得了知识和技能,也使我由衷地感受到天文学带来的快乐。

感谢同一课题组的任致远、王科、孙宁晨、刘博洋、王文慧、龙凤以及罗连通。他们对我的帮助和与我进行的讨论使我受益匪浅。

篇幅所限,在此对其余所有对本研究作出了贡献的个人和组织表示感谢。
